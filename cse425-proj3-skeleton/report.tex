% Created 2016-09-30 Fri 21:21
\documentclass[11pt]{article}
\usepackage[utf8]{inputenc}
\usepackage[T1]{fontenc}
\usepackage{fixltx2e}
\usepackage{graphicx}
\usepackage{grffile}
\usepackage{longtable}
\usepackage{wrapfig}
\usepackage{rotating}
\usepackage[normalem]{ulem}
\usepackage{amsmath}
\usepackage{textcomp}
\usepackage{amssymb}
\usepackage{capt-of}
\usepackage{hyperref}
\usepackage[top=0.25in, left=0.45in, right=0.75in, bottom=0.25in]{geometry}
\author{Nishant Gupta, nishgu@iitk.ac.in, 13447}
\date{\today}
\title{CS425 - Project 3 Report}
\hypersetup{
 pdfauthor={Nishant Gupta, nishgu@iitk.ac.in, 13447},
 pdftitle={CS425 - Project 3 Report},
 pdfkeywords={},
 pdfsubject={STCP protocol implementation},
 pdfcreator={Emacs 24.5.1 (Org mode 8.3.6)}, 
 pdflang={English}}
\begin{document}

\maketitle
\begin{center}
\texttt{\Huge STCP Protocol}
\end{center}
\newpage

\section{Implementation Specs}
\label{sec:orgheadline1}
\begin{itemize}
\item Sliding window is implemented
\item Proper endianness handling is insured
\item Abrupt disconnects are handled
\item Proper errno is set
\end{itemize}

\section{Test results}
\label{sec:orgheadline2}
\begin{itemize}
\item All Combinations of two clients(reference client and client.c in skeleton) and 
two server(reference server and server.c in skeleton) are working properly
\item The client exits cleanly with \texttt{Ctrl+D} and \texttt{Ctrl+C} interrupts and the server with \texttt{Ctrl+C} interrupt.
\item Single file transfer with \texttt{-f} flag is working properly
\item Multiple client connections are working fine with one client queued after other
\end{itemize}

\section{Appendix}
\label{sec:orgheadline4}
\subsection{Source code :}
\label{sec:orgheadline3}
\begin{verbatim}
/*
 * transport.c
 *
 *  Project 3
 *
 * This file implements the STCP layer that sits between the
 * mysocket and network layers. You are required to fill in the STCP
 * functionality in this file.
 *
 */

#include "transport.h"
#include "mysock.h"
#include "stcp_api.h"
#include <arpa/inet.h>
#include <assert.h>
#include <stdarg.h>
#include <stdio.h>
#include <stdlib.h>
#include <string.h>
#include <errno.h>

enum { CSTATE_ESTABLISHED }; /* you should have more states */

/* this structure is global to a mysocket descriptor */
typedef struct {
  bool_t done; /* TRUE once connection is closed */

  int connection_state; /* state of the connection (established, etc.) */
  tcp_seq initial_sequence_num;
  tcp_seq initial_sequence_num_peer;
  tcp_seq current_sequence_num_peer;
  tcp_seq current_sequence_num;
  tcp_seq sender_window, unacked, congestion_window,
      last_acked;
  /* any other connection-wide global variables go here */
} context_t;

static void generate_initial_seq_num(context_t *ctx);
static void control_loop(mysocket_t sd, context_t *ctx);


inline void hton(STCPHeader* header){
  header->th_seq = htonl(header->th_seq);
  header->th_ack = htonl(header->th_ack);
  header->th_win = htons(header->th_win);
}

inline void ntoh(STCPHeader* header){
  header->th_seq = ntohl(header->th_seq);
  header->th_ack = ntohl(header->th_ack);
  header->th_win = ntohs(header->th_win);
}
/* initialise the transport layer, and start the main loop, handling
 * any data from the peer or the application.  this function should not
 * return until the connection is closed.
 */
void transport_init(mysocket_t sd, bool_t is_active) {
  context_t *ctx;
  int bytes_recieved, send_status;
  ctx = (context_t *)calloc(1, sizeof(context_t));
  assert(ctx);

  generate_initial_seq_num(ctx);
  STCPHeader *hdr = (STCPHeader *)calloc(1, sizeof(STCPHeader));
  /* XXX: you should send a SYN packet here if is_active, or wait for one
   * to arrive if !is_active.  after the handshake completes, unblock the
   * application with stcp_unblock_application(sd).  you may also use
   * this to communicate an error condition back to the application, e.g.
   * if connection fails; to do so, just set errno appropriately (e.g. to
   * ECONNREFUSED, etc.) before calling the function.
   */
  if (is_active) {
    /**
    printf("Active Connection initiation request\n");
    */

    // send SYN packet
    hdr->th_flags = TH_SYN;
    hdr->th_seq = ctx->current_sequence_num++;
    hdr->th_off = sizeof(STCPHeader)/4;

    hton(hdr);
    send_status = stcp_network_send(sd, hdr, sizeof(STCPHeader), NULL);
    if(send_status == -1) {
      errno = ECONNREFUSED;
      return;
    }


    // recv and process SYN+ACK packet
    bytes_recieved = stcp_network_recv(sd, hdr, sizeof(STCPHeader));
    if(bytes_recieved == 0) {
      errno = ECONNREFUSED;
      return;
    }
    ntoh(hdr);
    assert(hdr->th_flags == (TH_SYN | TH_ACK));
    assert(ctx->current_sequence_num == hdr->th_ack);
    ctx->initial_sequence_num_peer = hdr->th_seq;
    ctx->current_sequence_num_peer = ctx->initial_sequence_num_peer + 1;

    // send ACK packet
    memset(hdr, 0, sizeof(STCPHeader));
    hdr->th_flags = TH_ACK;
    hdr->th_seq = ctx->current_sequence_num;
    hdr->th_off = sizeof(STCPHeader)/4;
    hdr->th_ack = ctx->current_sequence_num_peer;
    hton(hdr);
    send_status = stcp_network_send(sd, hdr, sizeof(STCPHeader), NULL);
    if(send_status == -1) {
      errno = ECONNABORTED;
      return;
    }


    // Handshake complete
  } else {
    /**
    printf("Passive Connection initiation request\n");
    */
    // recieve SYN packet
    bytes_recieved = stcp_network_recv(sd, hdr, sizeof(STCPHeader));
    if(bytes_recieved == 0) {
      errno = ECONNABORTED;
      return;
    }
    ntoh(hdr);
    ctx->initial_sequence_num_peer = hdr->th_seq;
    ctx->current_sequence_num_peer = ctx->initial_sequence_num_peer + 1;

    // send SYN+ACK packet
    memset(hdr, 0, sizeof(STCPHeader));
    hdr->th_flags = TH_ACK | TH_SYN;
    hdr->th_seq = ctx->current_sequence_num++;
    hdr->th_off = sizeof(STCPHeader)/4;
    hdr->th_ack = ctx->current_sequence_num_peer;
    hton(hdr);
    send_status = stcp_network_send(sd, hdr, sizeof(STCPHeader), NULL);
    if(send_status == -1) {
      errno = ECONNABORTED;
      return;
    }

    // recieve ACK
    stcp_network_recv(sd, hdr, sizeof(STCPHeader));
    if(bytes_recieved == 0) {
      errno =  ECONNABORTED;
      return;
    }
    ntoh(hdr);

    assert(hdr->th_flags & TH_ACK);
    assert(ctx->current_sequence_num == hdr->th_ack);
  }
  ctx->connection_state = CSTATE_ESTABLISHED;
  /**/
  //printf("Handshake Completed\n"); /**/
  stcp_unblock_application(sd);
  control_loop(sd, ctx);

  /* do any cleanup here */
  free(ctx);
}

/* generate random initial sequence number for an STCP connection */
static void generate_initial_seq_num(context_t *ctx) {
  assert(ctx);

#ifdef FIXED_INITNUM
  /* please don't change this! */
  ctx->initial_sequence_num = 1;
#else
  /* you have to fill this up */
  //  ctx->initial_sequence_num = 1; // testing

  ctx->initial_sequence_num = rand() % 256;
  ctx->current_sequence_num = ctx->initial_sequence_num;
#endif
}

/* control_loop() is the main STCP loop; it repeatedly waits for one of the
 * following to happen:
 *   - incoming data from the peer
 *   - new data from the application (via mywrite())
 *   - the socket to be closed (via myclose())
 *   - a timeout
 */
static void control_loop(mysocket_t sd, context_t *ctx) {
  assert(ctx);
  assert(!ctx->done);
  unsigned int event;
  bool_t this_end_closed = FALSE, other_end_closed = FALSE;
  STCPHeader *hdr = (STCPHeader *)calloc(1, sizeof(STCPHeader));
  int off = sizeof(STCPHeader) / 4 + ((sizeof(STCPHeader) % 4) ? 1 : 0);
  ctx->unacked = 0;
  ctx->sender_window = 3072;
  ctx->congestion_window = 3072;
  int send_status;
  void *buffer = (void *)malloc(STCP_MSS + 4 * off);
  while (!ctx->done) {

    event = stcp_wait_for_event(sd, ANY_EVENT, NULL);

    if ((event & APP_DATA) && !this_end_closed) {

      int data_limit = MIN(STCP_MSS, ctx->sender_window - ctx->unacked);

      hdr->th_seq = ctx->current_sequence_num;
      hdr->th_off = off;
      hdr->th_flags = 0;
      hdr->th_win = ctx->congestion_window;

      int bytes_to_send = stcp_app_recv(sd, buffer, data_limit);
      hton(hdr);
      send_status = stcp_network_send(sd, hdr, sizeof(STCPHeader), buffer, bytes_to_send,
                        NULL);
      if(send_status == -1) {
        errno = ECONNABORTED;
        return;
      }
      ctx->unacked += bytes_to_send;
      ctx->current_sequence_num += bytes_to_send;
    }

    if (event & NETWORK_DATA) {
      int bytes_received = stcp_network_recv(sd, buffer, STCP_MSS + off * 4);
      if(bytes_received == 0) {
        errno = ECONNABORTED;
        ctx->done = TRUE;
        continue;
      }

      memcpy((void *)hdr, buffer, sizeof(STCPHeader));
      ntoh(hdr);
      ctx->sender_window = MIN(hdr->th_win, ctx->congestion_window);

      if ((hdr->th_flags & TH_ACK) && ctx->last_acked < hdr->th_ack - 1) {
        ctx->last_acked = hdr->th_ack - 1;
        ctx->unacked = ctx->current_sequence_num - hdr->th_ack;
        ctx->sender_window = MIN(hdr->th_win, ctx->congestion_window);
        if(this_end_closed) { // we won't send any data after this
                              // so this should hold
          assert(hdr->th_ack <= ctx->current_sequence_num + 1);
        }
      }

      if (bytes_received - off * 4 > 0) {
        if(ctx->current_sequence_num_peer > hdr->th_seq + bytes_received - off * 4){
          // data is duplicate but send new ack
          hdr->th_ack = hdr->th_seq + bytes_received - off * 4;
          hdr->th_flags = TH_ACK;
          hdr->th_win = ctx->congestion_window;
          hdr->th_off = off;
          hdr->th_seq = ctx->current_sequence_num;

          hton(hdr);
          send_status = stcp_network_send(sd, hdr, off * 4, NULL);
        } else {
          int new_data_offset = 0;
          new_data_offset = ctx->current_sequence_num_peer - hdr->th_seq;
          ctx->current_sequence_num_peer = hdr->th_seq + bytes_received - off * 4;
          hdr->th_ack = ctx->current_sequence_num_peer;
          hdr->th_flags = TH_ACK;
          hdr->th_win = ctx->congestion_window;
          hdr->th_off = off;
          hdr->th_seq = ctx->current_sequence_num;

          hton(hdr);
          send_status = stcp_network_send(sd, hdr, off * 4, NULL);
          stcp_app_send(sd, (char *)buffer + off * 4 + new_data_offset, bytes_received - off * 4 - new_data_offset);
        }
        if(send_status == -1) {
          errno = ECONNABORTED;
          return;
        }
      }


      if (hdr->th_flags & TH_FIN) {
        ctx->current_sequence_num_peer++; // FIN takes one Segment space
        //ACK for FIN
        memset(hdr, 0, sizeof(STCPHeader));
        hdr->th_ack = ctx->current_sequence_num_peer;
        hdr->th_seq = ctx->current_sequence_num;
        hdr->th_flags = TH_ACK;
        hdr->th_win = ctx->congestion_window;
        hdr->th_off = off;

        hton(hdr);
        send_status = stcp_network_send(sd, hdr, off * 4, NULL);
        if(send_status == -1) {
          errno = ECONNABORTED;
          return;
        }
        other_end_closed = TRUE;
        stcp_fin_received(sd);
      }

    }


    if(event & APP_CLOSE_REQUESTED) {
      //printf("App close requested\n");
      // send FIN
      memset(hdr, 0, sizeof(STCPHeader));
      hdr->th_seq = ctx->current_sequence_num++;
      hdr->th_flags = TH_FIN;
      hdr->th_off = sizeof(STCPHeader)/4;
      hdr->th_win = ctx->congestion_window;

      hton(hdr);
      send_status = stcp_network_send(sd, hdr, off * 4, NULL);
      if(send_status == -1) {
        errno = ECONNABORTED;
        return;
      }


      this_end_closed = TRUE;
    }
    if (event & TIMEOUT) {
      ctx->done = TRUE;
    }

    if(this_end_closed && other_end_closed) ctx->done = TRUE;
    /* etc. */
  }
  free(hdr);
  free(buffer);
}

/**********************************************************************/
/* our_dprintf
 *
 * Send a formatted message to stdout.
 *
 * format               A printf-style format string.
 *
 * This function is equivalent to a printf, but may be
 * changed to log errors to a file if desired.
 *
 * Calls to this function are generated by the dprintf amd
 * dperror macros in transport.h
 */
void our_dprintf(const char *format, ...) {
  va_list argptr;
  char buffer[1024];

  assert(format);
  va_start(argptr, format);
  vsnprintf(buffer, sizeof(buffer), format, argptr);
  va_end(argptr);
  fputs(buffer, stdout);
  fflush(stdout);
}
\end{verbatim}
\end{document}